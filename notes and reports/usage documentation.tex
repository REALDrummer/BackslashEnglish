\documentclass{article}

\usepackage{indentfirst}		% indents the first line of paragraphs after (sub)*section headers
\usepackage[margin=1in]{geometry}	% makes 1in reasonable margins (as opposed to the huge default ones)
\usepackage{titlesec}			% allows the use of \paragraph and some other small section commands and tools
\usepackage[dvipsnames]{xcolor}	% allows the coloring of text, especially for the simulated syntax highlighting
\usepackage{lmodern}			% allows the use of lmodern fonts, which have less issues than the default like '<' doesn't become '¡' and tt text can be bold
\usepackage[T1]{fontenc}		% fixes those weird fint-compatibility issues that can turn '<' into '¡' and so forth

\setcounter{secnumdepth}{4}		% allows 4-point sequence numbers in delineating sections, e.g. "2.2.2.1"

\titleformat{\paragraph}
{\normalfont\normalsize\bfseries}{\theparagraph}{1em}{}
\titlespacing*{\paragraph}
{0pt}{3.25ex plus 1ex minus .2ex}{1.5ex plus .2ex}

\newcommand{\name}{\textbackslash{}English}				% just lets me avoid typing "\textbackslash{}English" all the time

% redefine headers for automatic labelling
\let\oldsection\section

\renewcommand{\section}[2]{\oldsection{#1}\label{#2}}

% commands for syntax highlighting code
\newcommand{\comment}[1]{\texttt{\textcolor{LimeGreen}{#1}}}
\newcommand{\type}[1]{\texttt{\textcolor{ForestGreen}{\textbf{#1}}}}
\newcommand{\keyop}[1]{\texttt{\textcolor{Purple}{\textbf{#1}}}}
\newcommand{\defpunct}[1]{\texttt{\textcolor{SkyBlue}{\textbf{#1}}}}
\newcommand{\define}[2]{\texttt{\keyop{define} \defpunct{<}#1\defpunct{>} #2\defpunct{:}}}
\newenvironment{code}[0]
{\ttfamily{}				% use true-type fonts for code
\setlength\parindent{0cm}	% remove paragraph indentation for code
~\\}
{\setlength\parindent{1cm}
~\\}

\begin{document}
\setlength\parindent{1cm}								% unify the paragraph indentation size

\tableofcontents
\newpage

\obeylines												% makes line breaks occur in the doducment as they occur in code

\section{Introduction: What, Why, and How?}{intro}
\subsection{What is \name{}?}

\subsection{What makes it different?}

\subsection{What is good about these differences?}

\subsection{How are these differences possible?}

\section{How to Use the Language}{how}
\subsection{Compilation}

\subsection{The Basic Basics}
\subsubsection{Comments}
\indent I mention comments first because they'll be used in future examples within this documentation.
\indent The octothorpe ('\#') begins single-line comments:
\begin{code}
some code		\comment{\# a comment}
\comment{\#\#\#\#\#\#\# comment comment comment}
more code		\comment{\#\# more comment with more \#s!}
\end{code}
\indent Octothorpes can be paired with angle brackets ('<' and '>') to make multi-line or inline comments:
\begin{code}
some \comment{\#< awesome >\#} code

\comment{\#\#\#< You can have as many \#s as you want and they don't
even have to match the number at the end! >\#\\}
code
\comment{\#< You can also have nested comments to make commenting out sections of code easier. 
I commented out the code below even though it also has multi-line comments.\\

\#< This is a comment >\# Here is some code that is commented and bad\\

Still, the comment doesn't end at the first ">" followed by an "\#"; it matches them like parentheses. 
>\#}
more code
\comment{\#\#< below is a pretty cool operator that we'll explain and even show you how to make later; this two-\# could be used as a doc comment (maybe?) >\#}
\define{\type{type of chars}}{string}
\qquad\type{chars}\defpunct{;}
\end{code}

\subsubsection{\texttt{main(}... Oh, Wait}
\indent There is no main method/function/operator. Just write what you want to do!

\subsubsection{Variables and Math}
\indent \name{} is strongly, statically typed. Variables need to be declared and need to be declared with a type. Variable declarations follow the same classic C-like format of a type followed by a name. Note that \name{} convention is the reverse of Java and many other languages: capitalize the first letter(s) of the variable names and keep types lowercase.

\begin{code}
\type{int} Integer
\type{queue} Queue
\end{code}

In \name{}, there's a bit of a twist: in this language, types, variable names, and other operators can have spaces, can be defined with special character-matching operators, and more. (See Section \ref{subsec:Operators} later on.)
\begin{code}
\type{int} Number of Variables
\type{list of queues of pairs of ints and floats} My Beautiful Queues
\end{code}

\paragraph{Standard Number Types}
\indent \name{} includes a wide variety of different types of numbers; some have static memory sizes and some don't. Here are some of the most basic number types that do \emph{not} have set memory sizes:
\begin{itemize}
	\item \type{number} is any kind of number, floating point or integer, static or arbitrary size, etc.
	\item \type{decimal} is any kind of rational number, including integers, floating-point numbers, or any special kinds of numbers like integral fractions or square roots calculated lazily. All \type{decimal}s are types of \type{number}s.
	\item \type{integer} is any kind of integer or any size, arbitrary or static. All \type{integer}s are types of \type{decimal}s.
\end{itemize}

\indent \name{} also includes a few types of numbers of static size:
\begin{itemize}
	\item \type{byte} is an 8-bit integer.
	\item \type{int} is a 32-bit integer.
	\item \type{float} is a 32-bit (single-precision) floating point number (IEEE-style).
\end{itemize}

\indent Other features of number types will be described in \ref{sssec:TypeModifiers}

\paragraph{Variable Declaration and Assignment}

\paragraph{Standard Arithmatic Operators}

\subsubsection{Lists and More}
\paragraph{Making a List}

\paragraph{Manipulating Lists By Index(es)}

\paragraph{Strings Are Lists And More!}

\subsection{Code Blocks}
\subsubsection{If and Its Friends}

\subsubsection{Loops}

\subsection{Operators} \label{subsec:Operators}
\subsubsection{Defining Them}

\subsubsection{Using Them}

\subsection{Types}
\subsubsection{Defining Them}

\subsubsection{Using Them}

\subsubsection{Type Modifiers}	\label{sssec:TypeModifiers}
\paragraph{Defining Them}

\paragraph{Using Them}

\subsubsection{Units}
\paragraph{Defining Them}

\paragraph{Using Them}

\subsection{Errors}
\subsubsection{Defining Them}

\subsubsection{Using Them}

\subsection{Input and Output}
\subsubsection{Stdin Reading and Stdout Printing}

\subsubsection{File Reading and Writing}

\subsection{Implicit Conversion}
\subsubsection{What?}

\subsubsection{Why?}

\subsubsection{Super-implicit?}

\subsection{High Order Functionality}
\subsubsection{What?}

\subsubsection{Why?}

\subsubsection{Missing Something? Partially Applied Operators}

\subsubsection{Operator Modifiers}

\subsubsection{Scope and Contextual Operators}

\section{The Basis of the Language}{basis}
\subsection{\texttt{\textbf{operator}}}

\subsection{\texttt{\textbf{define}}}

\subsection{\texttt{\textbf{asm}}}

\subsection{\texttt{\textbf{run}}}

\subsection{\texttt{\textbf{bits}}}

\subsection{\texttt{\textbf{use}}}

\section{The Not-So-Complicated Compilation}{compilation}
\subsection{Idea}

\subsection{Optimizability}
\end{document}
